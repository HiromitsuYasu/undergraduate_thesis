\chapter{\TeX のサンプル} \label{sec:sample}

\section{ラベルのはりかた}\label{sec:label}
label命令で名前をつける.\TeX では章節や図の番号は自動で振られるので,label命令でつけた名前にref命令でアクセスして番号を取得する.
例としてこの章は,\ref{sample}章,など.

\subsection{サンプル一覧}\label{sec:list}
ラベルの貼り方(\ref{sec:label}),数式のいれかた(\ref{sec:eq}),図のいれかた(\ref{sec:figure}),表のいれかた(\ref{sec:table}),参考文献の付け方(\ref{sec:ref})など.

\section{数式のいれかた}\label{sec:eq}
数式が美しく組版されるのは\TeX を使う主要な理由の一つである.\TeX の数式には行内に入れる数式(例$E=mc^2$)と別行立ての数式,

\begin{equation}
	E = \sqrt{m^2 c^4 +|\bm{p}|^2 c^2}
\end{equation}

がある.
数式番号を入れたくなければ,
\begin{equation}
	\nonumber
	E = \sqrt{m^2 c^4 +|\bm{p}|^2 c^2}
\end{equation}

複数行にわたる式で,位置を揃えたければ,
\begin{eqnarray}
	\nonumber
	E & =    & \sqrt{m^2 c^4 +|\bm{p}|^2 c^2} \\
	  & \sim & mc^2
\end{eqnarray}

とする.

ギリシャ文字を入れる場合、$\mu$のようにする.

\section{図のいれかた}\label{sec:figure}

一般に図はpdfファイル或いはjpgファイルで用意するのが良い.
pdfファイルでもjpgファイルでもいれかたは同じ.
図を入れたいところに入れようとしないこと.
適切な位置に挿入するのは\TeX 様がやってくれる.


\section{表のいれかた}\label{sec:table}

表は以下の書式で作成する.サンプルを表\ref{tab:test5}に示す.
\begin{table}[tb]
	\centering
	\caption{表の例}
		\label{tab:test5}	
	  \begin{tabular}{lcc} 
		\hline
		 		&信号& バックグラウンド \\ 
		\hline \hline
		例1 	& 1	 & 40,000			\\
		例2 	& 2  & 50,000			\\
		例3 	& 3  & 60,000			\\
		\hline
	  \end{tabular}
\end{table}

\section{引用のしかた}\label{sec:ref}
bibファイルをArXiVなりLead2Amazonなりからとってきて,reference.bibに取り込む.
本文中ではciteで自動的にリファレンスが作成される\cite{myThesisTest1}.
複数個つけることもできる\cite{myThesisTest2,myWebsite}.

