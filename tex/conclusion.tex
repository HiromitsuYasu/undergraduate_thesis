\chapter{結論}\label{cha:conclusion}
本研究の目的は光生成反応の飛跡を捉えることが目的であったが, うまく捉えることができなかった.
以下に示す2点の改良及び解析で光生成反応の飛跡を捉えることができるであろうと考えられる.

1つ目は装置のデザインの見直しである.本研究では納期や予算の都合上新しいシンチレータを用いることが難しい状況にあり, 既存のシンチレータを用いて光生成反応の検出に取り組む必要があった.その結果として棒状の長いシンチレータを斜めに交差させることで3次元飛跡検出器を作ることになった.この検出器は光生成反応を探索するには横方向の位置分解能が低いという問題点があるため, 横方向の位置分解能が良い検出器をデザインし直すことで光生成反応イベントの探索が容易になるであろうと考えられる.

2つ目はバックグラウンドの低減を図ることである.例えば、\ref{sec:anal:background}節で述べたようなパイオンと電子の角度分布の違いを利用して、実験装置と解析の両方において光生成反応によって生成されたパイオンが支配的な角度範囲に感度を持たせることなどが挙げられる。