\chapter{simulation}\label{simulation}
\section{Geant4}
Geant4 とは物質中を通過する粒子の物理相互作用をモンテカルロ法に基づいてシミュレートすることのできるパッケージである。
物理プロセスや検出器の構造、検出器の応答、応答データ等の作成、保存などの多くのツールキットから構成されている。

\section{本実験での$\mu$粒子シミュレーション}
宇宙線発生シミュレーションCRYを用いて$\mu$粒子を生成した。
装置のパラメーターは
    \begin{quote}
        \begin{itemize}
            \item トリガーシンチレータ126cm$\times$7cm$\times$1cmを横に3枚
            \item シンチレータ75cm$\times$4cm$\times$1cmを横に4枚、縦に8層
            \item アルミニウム板100cm$\times$30cm$\times$2cmを8層
        \end{itemize}
    \end{quote}
としてトリガーシンチレータのすぐ下からトリガーシンチレータの範囲で10万イベントの$\mu$粒子を発生させた。

\section{シミュレーション結果}
一番下のシンチレータまで到達したイベントは全イベントのうち1,937,095イベントで効率としては0.194であった。
また、シンチレータ内で反応して$\pi$が出たイベントは313イベントで効率は$3.13 \times10^{-5}$であった。
このことより、今回用いた検出器においても光生成反応が起きて$\pi$が出てくるイベントというのが観測できそうだということがわかった。
実際の測定器に$\mu$粒子が入射するレートは15count/secであったので、1秒に$4.70 \times10^{-4}$だけ$\pi$が出てくるイベントがあるということになる。

        
